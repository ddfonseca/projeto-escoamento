\section {Números adimensionais}

\subsection{Coeficiente de Reynolds}

    O número de Reynolds é bastante utilizado na mecânica dos fluidos, principalmente para determinar a classificação do escoamento, se é laminar, transiente ou turbulento. Seu significado físico é o quociente de forças de inércia ($V \rho$) entre forças de viscosidade ($\mu/D$), e é expressado como:

    \begin{equation} \label{eq-reynolds}
        Re = \frac{\rho V D}{\mu}
    \end{equation}

    Onde, $\rho$ é a massa específica do fluido, $V$ velocidade do fluido, $D$ o diâmetro da tubulação e $\mu$ a viscosidade cinemática do fluido.

\subsection {Coeficiente de forma}

    A equação abaixo representa a \textbf{velocidade de rotação específica} ou \textbf{coeficiente de forma} do rotor. Ela é adimensional e seu valor numérico mantém constante para máquinas de fluxo semelhantes, independente do sistema de unidades usado no cálculo.
    \begin{equation} \label{eq-na}
        n_{qA} = 10^3 \; . \; n \; . \; \frac{Q^{\rfrac{1}{2}}}{Y^{\rfrac{3}{4}}}
    \end{equation}

\noindent
    Onde:  \\
    $n_{qA} =$ velocidade de rotação específica ou coeficiente de forma do rotor; \\
    $n$ = velocidade de rotação da máquina, em $rps$ (Hz); \\
    $Q$ = vazão da máquina, em $m^3/s$; \\
    $Y$ = salto energético específico, em $J/kg$.

% \indent

    Os valores de $n$, $Q$, e $Y$, utilizados no cálculo do $n_{qA}$ correspondem ao ponto de projeto ou melhor rendimento. Para máquinas de vários estágios (rotores em série) o $Y$ utilizado corresponde ao salto energético específico de cada rotor, e para rotores com dupla sucção, utiliza-se a vazão $Q$ de um dos tubos de sucção.

    \begin{table}[htb]
         \setlength{\tabcolsep}{20pt} % General space between cols (6pt standard)
        \centering
        \caption{Valores de $n_{qA}$ indicados para diferentes tipos de máquinas de fluido.}
        \label{tab-na}
        % \begin{tabular}{p{2.6cm}|p{6.0cm}|p{2.25cm}|p{3.40cm}}
        % \resizebox{0.35\textwidth}{!}{%
        \begin{tabular}{l  l}
            \toprule
            Para turbina hidráulica do tipo Pelton & $n_{qA} =$ 5 a 70 \\
            \hline
            Para turbina hidráulica do tipo Francis lenta & $n_{qA} =$ 50 a 120 \\
            % \hline
            % Para turbina hidráulica do tipo Francis normal & $n_{qA} =$ 120 a 200 \\
            \hline
            Para turbina hidráulica do tipo Francis rápida & $n_{qA} =$ 200 a 320 \\
            \hline
            Para turbina hidráulica do tipo Michel-Banki  & $n_{qA} =$ 30 a 210 \\
            \hline
            Para turbina hidráulica do tipo Dériaz  & $n_{qA} =$ 200 a 450 \\
            \hline
            Para turbina hidráulica do tipo Kaplan e Hélice  & $n_{qA} =$ 300 a 1000 \\
            \hline
            Para turbina a vapor e gás com admissão parcial & $n_{qA} =$ 6 a 30 \\
            \hline
            Para turbina a vapor e gás com admissão total & $n_{qA} =$ 30 a 300 \\
            \hline
            Para bomba de deslocamento positivo & $n_{qA} <$  30 \\
            \hline
            Para bomba centrifuga & $n_{qA} =$  30 a 250 \\
            \hline
            Para compressor de deslocamento positivo & $n_{qA} <$ 20 \\
            \hline
            Para ventilador e turbocompressor centrífugo & $n_{qA} = 20$ a 330 \\
            \hline
            Para ventilador e turbocompressor axial & $n_{qA} =$ 330 a 1800 \\
            \bottomrule
        \end{tabular} %}
        \legend {Fonte: Máquinas de fluidos  \cite{maq_fluidos_henn}}
    \end{table}


\section{Perda de Carga}

    O assunto perda de carga é extenso dentro da mecânica dos fluidos, será tratado aqui apenas o essencial para o desenvolvimento do projeto. Sabe-se que entre dois pontos de um fluido é possível aplicar a equação abaixo, com as seguintes condições: escoamento permamente, incompreensível, completamente desenvolvido, energia interna e pressão uniforme em 1 e 2 \cite{fox}.

    \begin{equation} \label{eq-perda_de_carga}
        h_t = \left( \frac{p_1}{\rho} + \frac{V^2_1}{2} + gz_1 \right)- \left( \frac{p_2}{\rho} + \frac{V^2_2}{2} + gz_2 \right)
    \end{equation}

\noindent
    Onde: \\
    $h_t$ : Perda de carga total da tubulação entre os pontos 1 e 2. \\
    $p$   : Pressão nos pontos 1 e 2. \\
    $\rho$ : Massa específica do fluido. \\
    $z$ : Cota geométrica do ponto 1 e 2. \\
    $V$: Velocidade nos pontos 1 e 2.


    Para um escoamento laminar, é possível calcular a ($h_t$) analiticamente, porém tubulação desenvolvida para o projeto, o escoamento é turbulento e o cálculo da perda de carga ($h_t$) é encontrado por uma fórmula empírica em base dados experimentais, conhecida por Darcy-Weisbach:
    \begin{equation} \label{eq-darcy}
        h_{maior} = f \frac{L}{D} \frac{V^2}{2}
    \end{equation}

    \noindent
    Onde \\
    $f$: fator de atrito \\
    $L$: Comprimento do duto \\
    $D$: Diâmetro interno do duto \\
    $V$: Velocidade no duto

    E $h_{maior}$ é denominado perda de carga maior, pois é onde terá uma maior contribuição para a perda de carga total, as perdas menores são devido uma variedade de acessórios, curvas ou mudanças súbitas de área \cite{fox} que não serão consideradas na tubulação do projeto.

    O fator de atrito $f$ pode ser calculado através da fórmula de Colebrook \cite{fox}, que serve para evitar o uso do Diagrama de Moody nos escoamentos turbulentos:

    \begin{equation} \label{eq-colebrook}
    \frac{1}{f^{0,5}} = -2,0 \log \left( \frac{e/D}{3,7} + \frac{2,51}{Re\;f^{0,5}} \right)
    \end{equation}

    Que é facilmente resolvida dando-lhe um valor inicial arbitrário  e fazendo a iteração dos novos valores, pois converge rapidamente.

\section{Máquinas de fluxo semelhantes}

    Há diversas vantagens em se construir máquinas de fluxo semelhantes, enquanto para modelos reduzidos permite diminuir a execução errônea de máquinas de grande porte, a construção de modelos aumentados muitas vezes se faz necessária para facilitar as medições durante os ensaios \cite{maq_fluidos_henn}.

    Para que duas máquinas sejam consideradas semelhantes, é necessário satisfazer algumas condições, como semelhanças geométricas, cinemáticas e dinâmicas.

    A \textbf{semelhança geométrica} significa que as duas máquinas devem ter as mesmas dimensões lineares, igualdade de ângulos e nenhuma omissão ou adição de partes. Para uma máquina de fluxo modelo (índice ``m'') e máquina protótipo (índice ``p'') sejam semelhantes geometricamente, é necessário que:

    \begin{equation} \label{eq-geometrica}
        \frac{D_{5p}}{D_{5m}} = \frac{b_{5p}}{b_{5m}} = \frac{D_{4p}}{D_{4m}} = k_G = \mbox{constante}
    \end{equation}

    Onde $k_G$ é denominado escala geométrica ou \textbf{fator de escala} e os ângulos de entrada e saída do protótipo e modelo também necessitam ser iguais, logo:

    \begin{equation}
        \beta_{4p} = \beta_{4m} \; \; \; \mbox{e} \; \; \; \beta_{5p} = \beta_{5m}
    \end{equation}

    Já a \textbf{semelhança cinemática} significa que as velocidades e acelerações, para pontos correspondentes, sejam vetores paralelos e possuam relação constante entre seus módulos, ou seja:

    \begin{equation}
    \frac{c_{m4p}}{c_{m4m}} = \frac{c_{u5p}}{c_{u5m}} = \frac{u_{5p}}{u_{5m}} = k_c = \mbox{constante}     \end{equation}

    Onde $k_c$ é denominado \textbf{escala de velocidades}. E a \textbf{semelhança dinâmica} significa que os tipos idênticos de forças sejam vetores paralelos e que a relação entre seus módulos seja constante para pontos correspondentes, ou seja:

    \begin{equation}
        \frac{F_{inércia,p}}{F_{inércia,m}} = \frac{F_{atrito,p}}{F_{atrito,m}} = k_D = \mbox{constante}
    \end{equation}

    Onde $k_D$ é denominada de \textbf{escala dinâmica}.

    É possível testar se duas máquinas realmente são semelhantes se cumprirem-se, simultaneamente, a igualdade do número de reynolds, do número de Mach, do número de Froude, do número Weber e do número de Euler. Contudo, o mais importante é o número de Reynolds para a semelhança dinâmica.

     As grandezas biunitárias são feitas restrições de salto energético específico unitário, diâmetro característico do rotor unitário, semelhança cinemática e rendimento hidráulico constante entre modelo e protótipo, A \textbf{velocidade de rotação biunitária} ($n_{11}$) é dada por:

     \begin{equation} \label{eq-vel-bi}
         n_{11} = \frac{n\: D}{Y^{\rfrac{1}{2}}}
    \end{equation}

    \noindent
    Onde: \\
    $n_{11}$ = velocidade de rotação biunitária, no Sistema Internacional de Unidades, em $kg^{1/2}\: m/J^{1/2}s$; \\
    $n$ = velocidade de rotação da máquina considerada, em rps ou Hz; \\
    $D$ = diâmetro característico do rotor da máquina considerada; \\
    $Y$ = salto energético da máquina considerada, em $J/kg$.

    Outra grandeza biunitária bastante utilizada em máquinas de fluxo, é a \textbf{potência no eixo biunitária} que é dada por:

    \begin{equation} \label{eq-pot-bi}
        P_{e11} = \frac{P_e}{D^2 \: Y^{3/2}}
    \end{equation}

